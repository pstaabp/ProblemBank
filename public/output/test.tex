\documentclass{article}

\usepackage[total={7in,9.5in}]{geometry}
\usepackage{amsmath}

\providecommand{\tightlist}{%
  \setlength{\itemsep}{0pt}\setlength{\parskip}{0pt}}
\begin{document}

\begin{enumerate}
\item Each of the following describes a line. Find the slope-intercept form of
the line with each of the following properties:

\begin{enumerate}
\def\labelenumi{\arabic{enumi}.}
\item
  The line has slope 3 and passes through the point \((-2,4)\).
\item
  The line passes through the points \((5,2)\) and \((-3,4)\).
\end{enumerate}
\item The cost \(C\), in dollars, of renting a moving truck for a day is
modeled by the function \(C(x)=0.25x+35,\) where \(x\) is the number of
miles driven.

\begin{enumerate}
\def\labelenumi{\arabic{enumi}.}
\tightlist
\item
  What is the cost if you drive \(x=40\) miles?
\item
  If the cost of renting the moving truck is \$80, how many miles did
  you drive?
\item
  Suppose that you want the cost to be no more than \$100. What is the
  maximum number of miles that you can drive?
\item
  What is the the implied domain of \(C\)?
\end{enumerate}
\item Plot each of the lines on the following axes. You should use knowledge
of the \(y\)-intercept and the slope for this. Do not just plot points.
Label each line (a)-(d).

\begin{enumerate}
\def\labelenumi{\arabic{enumi}.}
\tightlist
\item
  \(y=2x-3\)
\item
  \(y=-\frac{2}{3}x + 4 \)
\item
  \(y=4\)
\item
  \(x=-1\)
\end{enumerate}

\end{enumerate}

\end{document}
