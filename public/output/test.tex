\documentclass{article}

\usepackage[total={7in,9.5in}]{geometry}
\usepackage{amsmath}

\providecommand{\tightlist}{%
  \setlength{\itemsep}{3pt}\setlength{\parskip}{3pt}}
\begin{document}


\noindent
\begin{tabular*}{\textwidth}{@{\extracolsep{\fill}}lcr}
& Fitchburg State University & \\
\makebox[2in][l]{Precalculus}  & Department of Mathematics & \makebox[2in][r]{June 30, 2016}\\
& Linear Functions HW
\end{tabular*}

\noindent\hrulefill

This is the homework for the Linear Functions modules.  You should write up your solutions according to the guidelines for written homework and submit it in class before July 17, 2016.  


\noindent\hrulefill


\begin{enumerate}
\item Each of the following describes a line. Find the slope-intercept form of
the line with each of the following properties:

\begin{enumerate}
\def\labelenumi{\arabic{enumi}.}
\item
  The line has slope 3 and passes through the point \((-2,4)\).
\item
  The line passes through the points \((5,2)\) and \((-3,4)\).
\end{enumerate}
\hrulefill 

\begin{enumerate}
\def\labelenumi{\arabic{enumi}.}
\tightlist
\item
  Using the point-slope form of the line, \(y=m(x-x_0)+y_0\) with
  \(m=3\) and the point \((-2,4)\), we get
  \[ y=3(x-(-2))+4 = 3(x+2)+4 = 3x+6+4\] so the slope-intercept form of
  the line is \(y=3x+10\).\\
\item
  First, find the slope through the points \((5,2)\) and \((-3,4)\).
  \[ m = \frac{2-4}{5-(-3)} = \frac{-2}{8} = -\frac{1}{4}\] Then using
  the point-slope form of the line with the first point,
  \[ y=-\frac{1}{4}(x-5)+2 = - \frac{x}{4}+\frac{5}{4}+2 \] so the
  slope-intercept form of the line is \[ y=-\frac{x}{4} + \frac{13}{4}\]
\end{enumerate}


 \hrulefill\item The cost \(C\), in dollars, of renting a moving truck for a day is
modeled by the function \(C(x)=0.25x+35,\) where \(x\) is the number of
miles driven.

\begin{enumerate}
\def\labelenumi{\arabic{enumi}.}
\tightlist
\item
  What is the cost if you drive \(x=40\) miles?
\item
  If the cost of renting the moving truck is \$80, how many miles did
  you drive?
\item
  Suppose that you want the cost to be no more than \$100. What is the
  maximum number of miles that you can drive?
\item
  What is the the implied domain of \(C\)?
\end{enumerate}
\hrulefill 

\begin{enumerate}
\def\labelenumi{\arabic{enumi}.}
\tightlist
\item
  Since \(C(40)=0.25(40)+35 = 45\), the cost is \$45.
\item
  Solve \(C(x)=80\), \[0.25x+35=80\] \[0.25x=45\]
  \[x = \frac{45}{0.25}=180\] so 180 miles were driven.
\item
  In this case, we solve \(C(x)<100\) \[0.25x+35<100\] \[0.25x<65\]
  \[x< \frac{65}{0.25} = 260\] So to keep the cost under \$100, you must
  drive less than 260 miles.
\item
  The values of \(x\) that make sense for this problem is only positive
  miles can be driven, so the domain is \((0,\infty)\).
\end{enumerate}


 \hrulefill\item Plot each of the lines on the following axes. You should use knowledge
of the \(y\)-intercept and the slope for this. Do not just plot points.
Label each line (a)-(d).

\begin{enumerate}
\def\labelenumi{\arabic{enumi}.}
\tightlist
\item
  \(y=2x-3\)
\item
  \(y=-\frac{2}{3}x + 4 \)
\item
  \(y=4\)
\item
  \(x=-1\)
\end{enumerate}
\hrulefill 

Need a solution for this.


 \hrulefill

\end{enumerate}

\end{document}
